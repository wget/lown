\documentclass{article}
\usepackage[top=2cm, bottom=2cm, left=2cm, right=2cm]{geometry}
\usepackage{hyperref}

% Add periods after title numbers (chapters and sections and in table of contents)
\usepackage{titlesec}
\titleformat{\section}{\bfseries\large}{\thesection}{.5em}{}[\hrule]
\renewcommand{\thesection}{\thechapter\arabic{section}.}
\renewcommand{\thesubsection}{\thesection\arabic{subsection}.}

% As we won't use paragraph{} statement, we are gonna have to redefine the
% inter-line space.
\usepackage{setspace}
%\singlespacing
\onehalfspacing

% Indent the beginning of every section
\usepackage{indentfirst}

% Ask to have space between 2 paragraphs.
\usepackage{parskip}

\usepackage{listings}
\lstset{
    basicstyle=\normalsize\ttfamily,
    frame=single,
    columns=flexible,
    breaklines=true,
    % Used because the Vim LaTeX colorscheme is buggy, using a $ character will
    % break the colorscheme. Defining a <dollar> statement avoids to use
    % a $ character.
    literate={<dollar>}{\$}1
}

\begin{document}

\title{\underline{L}ibre\underline{O}ffice \underline{W}eekly \underline{N}ews \#2}
\author{William Gathoye}
\date{June 20, 2014}
\maketitle



\emph{LibreOffice Weekly News consists of a weekly issue, published at the end of
each week, gathering interesting development reported in discussions (mailing
lists or IRC), Q/A and git commits made over the past week. This summary will
emphase on the developer point of view.
This week in LibreOffice\ldots}



\section{GUI.}

The conversion towards the .ui glade dialogs is still continuing (thanks Mihály Palenik!) \cite{gladeUi1,gladeUi2,gladeUi3}.

Do not forget wehave a GSoC student, in charge of the dialog widget conversion too\cite{gsocDialogConversion}. LibreOffice will use even more .ui dilaog in the future. Thanks Szymon Kłos.


It has been discussed to add the ability to select the color of non printable characters since the light blue color which is currently used is too difficult to be read when using a white page (the most common configuration)\cite{lightBlueCharacter}. Since LibreOffice 4.3 UI is freezed, it should be added to 4.4, and definitely considered as an easy\cite{lightBlueCharacterEasyHackProposal}.

\section{License and FOSS.}

LibreOffice has been mentioned in Fortune\cite{fortuneTime}, the business magazine published by Time Inc, editor of the famous TIME magazine. A LibreOffice contributor reacted on the mailing list\cite{fortuneTimeReaction}, and precised that patent and copyrights are 2 different concepts: stating that even if LibreOffice is a FOSS like any similar projects, it is still copyrighted, but not patented. 


\section{Code base evolution.}

Fix the JDK dependency on OS X for Rhino\cite{javaDepRhino1}\cite{javaDepRhino2}, a FOSS implementation of Javascript written in Java\cite{rhinoDefinition} (thanks Robert Antoni Buj i Gelonch). JDK > 1.6 and JDK 1.8 can now be used on OS X.

If you wonder where such a Javascript implementation would be needed: for UNO bridges. See this article to know more about UNO bridges\cite{unoBridges}.


Last week, we spoke about a copy paste code detector. Now we are beginning to have some great results from that tool. Duplicated code has been found and will be merged\cite{duplicateFound}.


\section{Build environment.}




\section{Unit tests.}




\section{Other issues.}

An user has reported that LibreOffice couldn't work with DEVONThink, a FOSS document organizer and previewer for OS X, while other old branches of LibreOffice (OO.o and NeoOffice) were able to do so. Is seems like the reason comes from the removal of a quicklook plugin into LibreOffice. The seldom bugs found Bugzilla don't come in handy this time\cite{quicklookPlugin}

Florian Reisinger wrote a trivial document explaining why assertions (which could be used in unit tests) are so much important to avoid and detect bugs. He gave the example of a linked list. To manipulate this list, we have methods like \lstinline{addElement}. Three conditions are playing a role in a method: the precondition (the input), the postcondition (output) but in this case mostly invariance. Invariance is  typically what should be valid at the beginning and ant the end of the method. These 2 locations will be used to put an \lstinline{assert} and detect if an error occured. In this example, if the linked list had 9 elements, after we called the method, the list should have still 9 elements (if the argument was invalid for example; in Object-Oriented Programming, object's methods must check inputs to protect the object integrity, see encapsulation) or more element, otherwise, something goes wrong\cite{assertionsUnitTest}.

\section{GSoC.}

Improvements to the Template manager (Efe Gürkan YALAMAN)\cite{gsocTemplateManager}.


\subsection{OLD content.}

Refactor god objects (Valentin): weekly report $\rightarrow$ pushed to master
and IDocumentChartDataProviderAccess has been nearly refactored.

Improve Usability of Personas (Rachit Gupta): the user can select the theme he
wants from the gallery and can be applied to the UI.  Needs to be done: run the
application of the theme in a separate thread to avoid the UI from hanging
+ little changes to the UI.

Connection to SharePoint and Microsoft OneDrive (Mihai Varga): all OneDrive is
serialized, a file and folder UI representation still needs to be writen. Now
writing unit tests.

Calc / Impress tiled rendering support (Andrzej Hunt): renamed to
LibreOfficeKit. For more details about the current implementation, please read
\cite{tiledRenderingArticle}.

As a reminder, tiled rendering is the ability to be able to paint (render)
a specific part of a document (a tile) to anywhere. This is really useful to
plug LibreOffice into various new scenarios where the traditional LO UI cannot
be used (mobile devices,\ldots).



\section{Hackfests and conferences.}

After the hackfest from Paris (Montreuil) hold on June 27-28th (thanks
Charles-H Schultz and Sophie Gautier), Toulouse has been confirmed to hold
a hackfest on November 15-16th\cite{hackfestToulouse} (thanks Arnaud Versini).
More to come at \cite{hackfestToulouseWiki}.

In 2015 linux.conf.au will be held 12-16 January in Auckland, New Zealand, at
the University of Auckland Business School.  Call for Proposals opened 9 June
2014, closes 13 July 2014\cite{linuxConfAuckland1}. There will not have
a LibreOffice booth during the conference itself, only on the open
day\cite{linuxConfAuckland2}.



\begin{thebibliography}{30}

\bibitem{wgetGithub}
\url{https://github.com/wget/lown}

\bibitem{quicklookPlugin}
    \url{http://listarchives.libreoffice.org/global/users/msg39426.html}

\bibitem{lightBlueCharacter}
    \url{http://listarchives.libreoffice.org/global/design/msg06643.html}

\bibitem{lightBlueCharacterEasyHackProposal}
    \url{http://listarchives.libreoffice.org/global/design/msg06646.html}

\bibitem{rhinoDefinition}
    \url{https://developer.mozilla.org/en-US/docs/Mozilla/Projects/Rhino}

\bibitem{javaDepRhino1}
    \url{https://gerrit.libreoffice.org/#/c/9784/}

\bibitem{javaDepRhino2}
    \url{https://gerrit.libreoffice.org/#/c/9785/}

\bibitem{unoBridges}
    \url{https://wiki.openoffice.org/wiki/Uno/Article/About_Bridges}

\bibitem{gladeUi1}
    \url{https://gerrit.libreoffice.org/#/c/9696/}

\bibitem{gladeUi2}
    \url{https://gerrit.libreoffice.org/#/c/9696/}

\bibitem{gladeUi3}
    \url{https://gerrit.libreoffice.org/#/c/9779/}

\bibitem{gsocDialogConversion}
    \url{https://www.google-melange.com/gsoc/project/details/google/gsoc2014/sk94/5685265389584384}

\bibitem{fortuneTime}
    \url{http://fortune.com/2014/06/13/the-one-asterisk-on-teslas-patent-giveaway/}

\bibitem{fortuneTimeReaction}
    \url{http://listarchives.libreoffice.org/global/marketing/msg16993.html}

\bibitem{duplicateFound}
    \url{https://bugs.freedesktop.org/show_bug.cgi?id=39593#c18}

\bibitem{gsocTemplateManager}
    \url{http://lists.freedesktop.org/archives/libreoffice/2014-June/061809.html}

\bibitem{assertionUnitTest}
    \url{http://flosmind.wordpress.com/2014/06/15/assertion-errors-not-only-for-devs/}

\end{thebibliography}

\end{document}
